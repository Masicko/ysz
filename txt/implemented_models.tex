\documentclass{article}
\usepackage[utf8]{inputenc}
\usepackage[usenames,dvipsnames]{xcolor} %monochrome option to ignore red, blue etc
\usepackage{amsmath,amssymb,graphicx, cancel, supertabular, booktabs, mathtools}
\usepackage{tikz}
\usepackage{siunitx, mhchem, comment}
%, stmaryrd}
\usepackage{bm}         % bold math fonts with \bm{}
\usepackage[font=small,labelfont=bf]{caption} % Required for specifying captions to tables and figures
\usepackage{pgfplotstable}
\usepackage{pgfplots}
\pgfplotsset{compat=1.9}
%\usepackage{lipsum}
%\frenchspacing
% bibliography
%\usepackage[backend=biber, style=numeric, bibencoding=UTF8]{biblatex}	% style=iso-numeric
%\bibliography{library.bib} 
%\journal{}
% My own styles package...
%\usepackage{customstyle}
\usepackage{wrapfig, subcaption, caption}
\usepackage{xargs}                      % Use more than one optional parameter in a new commands
\usepackage[colorinlistoftodos,prependcaption,textsize=tiny]{todonotes}
\newcommandx{\unsure}[2][1=]{\todo[linecolor=red,backgroundcolor=red!25,bordercolor=red,#1]{#2}}
\newcommandx{\change}[2][1=]{\todo[linecolor=blue,backgroundcolor=blue!25,bordercolor=blue,#1]{#2}}
\newcommandx{\info}[2][1=]{\todo[linecolor=OliveGreen,backgroundcolor=OliveGreen!25,bordercolor=OliveGreen,#1]{#2}}
\newcommandx{\improvement}[2][1=]{\todo[linecolor=Plum,backgroundcolor=Plum!25,bordercolor=Plum,#1]{#2}}
\newcommandx{\thiswillnotshow}[2][1=]{\todo[disable,#1]{#2}}

\numberwithin{equation}{section}

\usepackage[normalem]{ulem}
\usepackage[draft]{showkeys}

\definecolor{colorRed}{rgb}{0.67,0.,0.}
\definecolor{colorBlue}{rgb}{0.,0.,0.67}

\newcommand{\BR}{\color{colorRed}}
\newcommand{\ER}{\color{black}}
\newcommand{\BB}{\color{colorBlue}}
\newcommand{\EB}{\color{black}}

%\newcommand{\ara}[1]{\renewcommand{\arraystretch}{#1}}
\newcommand{\us}[1]{\underset{\textrm{s}}{#1}{}}

\def\tilde{\widetilde}
\def\cI{\mathcal{I}}
\def\F{\textrm{F}}
\def\e{\textrm{e}}
\def\Ref{\mathit{ref}}

\def\kB{k_\mathrm{B}}

\def\Ox{\mathrm{O}}
\def\oo{{\ce{O2}}}
\def\nn{{\ce{N2}}}
\def\Om{\mathrm{Om}}
\def\Oi{\mathrm{Oi}}
\def\Zr{\mathrm{Zr}}
\def\Yi{\mathrm{Y }}
\def\OO{\mathrm{O}_2}

\def\Mp{\mathrm{M^+}}
\def\eM{\mathrm{e^-}}


\def\MLC{M_\mathrm{C}^\mathrm{L}}
\def\MLA{M_\mathrm{A}^\mathrm{L}}
\def\nLA{n_\mathrm{A}^\mathrm{L}}
\def\nLC{n_\mathrm{C}^\mathrm{L}}
\def\nC{n_\mathrm{C}}
\def\nG{n^\mathrm{g}}
\def\nL{n_\mathrm{L}}
\def\zL{z_\mathrm{L}}
\def\zA{z_\mathrm{A}}
\def\mL{m_\mathrm{L}}
\def\vL{v_\mathrm{L}}
\def\aL{a_\mathrm{L}}

\def\ys{\us y}
\def\y0{y(0)}
\def\yOs{\us y_\textrm{O}}
\def\pO2{p\OO}

\def\A0{A_0}
\def\R0{R_0}
\def\K0{O_0}
\def\DGA{\Delta G_\textrm{A}  }
\def\DGR{\Delta G_\textrm{R}  }
\def\DGO{\Delta G_\textrm{O}  }
\def\betaA{\beta_A}
\def\betaR{\beta_R}
\def\betaO{\beta_O}
\def\SA{S_A}
\def\SR{S_R}
\def\SO{S_O}


\def\eq{\textrm{eq}}
\def\REF{\textrm{ref}}

\def \YSZ{\textrm{YSZ}}
\def \M{\textrm{M}}

\def \yY{y^\YSZ}
\def \varphiY{\varphi^\YSZ}
\def \neM{n_\textrm{e}^\M}
\def \varphiM{\varphi^\M}

\def \yYs{{\us y}^\YSZ}
\def \neMs{\us \neM}

\def\pD{\tilde{D}}
\def\PF{P}


\def\vau{{\BB\upsilon\EB}}

%
%Metadata
\usepackage{hyperref}
\hypersetup{pdftitle=Yttria-stabilized Zirconia}
\hypersetup{pdfauthor=Vojtech Milos}
\begin{document}
%\begin{frontmatter}
\author{Vojtěch Miloš, Petr Vágner, Jürgen Fuhrmann}
\title{Implemented models}
\maketitle
%\address{Charles University, Weierstraß Institute}
%\end{frontmatter}

%\section{YSZ characterization}
%Exists in different crystalline forms, single crystal and polycrystalline shown in figures %\ref{fig:phase_d},\ref{fig:cryst}.
%For further details consult \cite{ikeda1985electrical, viazzi2008structural}.
%%\begin{figure}[h] 
%%\includegraphics[width=0.5\textwidth]{./img/phase_diagram_Scott.png}
%%\caption{}
%%\label{fig:phase_d}
%%\end{figure}
%%\begin{figure}[h] 
%%\includegraphics[width=0.7\textwidth]{./img/YSZ_crystalline_phases.png}
%%\caption{}
%%\label{fig:cryst}
%%\end{figure}
%\subsection{YSZ thermal expansion}
%\cite{hayashi2005thermal}
%Linear thermal expansion coefficient $\frac{\alpha_\textrm{V}}{3} =\alpha_\textrm{L} = \frac{1}{L}\frac{\did L}{\did T} \approx \SI{1e-6}{\per\kelvin}\text{	at }\SI{1000}{\kelvin}$.
%Grüneisen's constant $\gamma \approx 1.4$, bulk modulus $K_\textrm{T} \approx \SI{205}{\giga\pascal}@\SI{298}{\kelvin}$
%\subsection{YSZ ionic conductivity}
%The main factor driving factor is definitely temperature.
%The other factors steering the YSZ ionic conductivity are the mixing ratio $x$, crystalline structure and atmosphere.
%\subsubsection{\citeauthor{bauerle1969study}~\cite{bauerle1969study}}
%
%\subsubsection{\citeauthor{fergus2006electrolytes}~\cite{fergus2006electrolytes}, review}
%Conductivity od YSZ increases up to $8$ mole$\%$ and then decreases. The decrease at higher dopant contents might
%be due to association of point defects, which leads to reduction in defect mobility and thus conductivity.
%Size mismatch between $\ce{Zr}$ and $\ce{Y}$ leads to lower conductivity greater energy for defect association.
%
%Grain boundary conduction is said to be important.
%\begin{figure}[h]
%%\includegraphics[width=\linewidth]{./img/recherche/fergus2006_fig2.png}
%\caption{YSZ conductivities taken from~\cite{fergus2006electrolytes}, 
%apparently the value is around $\SIF{0.1}{\siemens\per\cm}$ at $\SIF{1000}{\celsius}$. }
%\end{figure}
%%
%\subsubsection{\citeauthor{chen2002influence}~\cite{chen2002influence}, granular polycrystal}
%Authors use fine-granular 8YSZ, measures EIS at temperature between $\SIF{300}{\celsius}$ and $\SIF{1000}{\celsius}$.
%Authors distinguish between intra- and intergranular conductivity, which they obtain by fitting EIS to an equivalent circuit.
%Specimen coated with a platinum paste a connected with platinum wire. Presumably air atmosphere was used.
%
%My opinion is that two different spectra are measured at high and low temperatures.
%Reported total conductivity was around $\SIF{0.1}{\siemens\per\cm}$ at
%$\SIF{1000}{\celsius}$.
%\begin{figure}[h]
%	\centering
%	%\includegraphics[width=\linewidth]{./img/recherche/chen2002_fig1.png}
%	\caption{Typical impedance spectra of YSZ at low temperature from \cite{chen2002influence}.}
%	\label{fig:chen2002}
%\end{figure}
%\subsubsection{\citeauthor{zhang2007ionic}~\cite{zhang2007ionic}}
%\begin{figure}[h]
%	\centering
%	%\includegraphics[width=\linewidth]{./img/recherche/zhang2007_fig2.png}
%	\caption{The equivalent circuit}
%	\label{fig:zhang2007}
%\end{figure}
%\subsubsection{\citeauthor{velle1991electrode}~\cite{velle1991electrode}}
%
%%\subsubsection{\citeauthor{manning1997kinetics}~\cite{manning1997kinetics}}
%Manning used $\langle 100\rangle$ single crystal 9.5YSZ and reported $\SIF{3e-2}{\siemens\per\cm}$ in $\SI{1}{\bar}$ air at $\SIF{800}{\celsius}$
%and slightly lower values in oxygen-free nitrogen.
%\vfill\eject

%%%%%%%%%%%%%%%%%%%%%%%%%%%%%%%%%%%%%%%%%%%%%%%%%%%%%%%%%%%%%%%%%%%%%%%%%%%%%%%%%%%%%




\section{Summary of the published model}
A proper summary of the model is nesseccary for having good time during implementation and during the whole life. 

\subsection{Model with dimensions}
The finall model equations for a half-cell reads

\begin{subequations}
\begin{align}
\partial_t \frac{m_\Ox m (1- \nu)}{\vL} y
+
\partial_x 
\left(
	\left(
		1
		+
		\frac{m_\Ox}{\mL} m (1 - \nu) y
	\right)
	\bm J_\Ox
\right)
&= 
0,
\\
- \varepsilon_0 (1 + \chi) \partial_{xx} \varphi 
&= 
\frac{e_0}{\vL}
\left(
	z_\textrm{A} m (1 - \nu) y
	+
	\zL
\right),
\\
\bm J_\Ox 
=
- \frac{D \ m_\Ox m (1 - \nu)}{\vL}
\left(
	1
	+
	\frac{m_\Ox}{\mL} m (1 - \nu) y
\right)
&\Bigg[
	\frac	
	{\partial_x y}
	{(1- y)}
	+ 
	z_\textrm{A} y \frac{e_0}{\kB T}
	\partial_x \varphi
\Bigg]
\end{align}
\end{subequations}
for bulk with the choise of mobility coefficient 
$$M = D \frac{m_\Ox}{\kB} \rho_\Ox = D \frac{m_\Ox^2 m (1 - \nu) y}{\vL \ \kB},$$ 
where $[D] = m^2 s^{-1}$ is a diffusion coefficient, and
\begin{subequations}
\begin{align}
\partial_t &\frac{m_\Ox \us m (1- \us \nu)}{\aL} \us y
-
m_\Ox \us A_0 
\left(
	\frac{\DGA}{\kB T}
	+
	\ln \frac{y(1- \us y)}{\us y (1- y)}
\right)
=
m_\Ox \us R,
\\ \nonumber
\\
&\us R
=
\us R_0
\Bigg[
	\textrm{exp}
	\left(
		{-\beta A \frac{\DGR}{\kB T}}
	\right)
	\left(
		\frac{\us y}{1- \us y}
	\right)^{-\beta A}
	{p_\Ox}^{\frac{\beta A}{2}}
\\ \nonumber
	&\quad \quad \quad \quad \quad \quad	
	-
	\textrm{exp}
	\left(
		{(1-\beta) A \frac{\DGR}{\kB T}}
	\right)
	\left(
		\frac{\us y}{1- \us y}
	\right)^{(1-\beta) A}
	{p_\Ox}^{-\frac{(1-\beta) A}{2}}
\Bigg],
\end{align}
\end{subequations}
for surface with the choice of adsorbtion coefficient
$$
\us M = \us A_0  \frac{m_\Ox^2}{\kB},
$$
where $[\us A_0] = \textrm{m}^{-2} \textrm{s}^{-1}$ denotes the rate of adsorbtion. The electrochemical reaction is supposed to be
\begin{align}
\ce{\frac{1}{2} \Ox_2 + 2\textrm{e}^- <=> \Ox^{2-}}
\end{align}
and we define
\begin{subequations}
\begin{align}
\DGA
&=
m_\Ox
\left(
	\psi_{\textrm{Om}}^\REF
	-
	\us \psi_{\textrm{Om}}^\REF
\right)
\\
\DGR
&= 
m_\Ox
\left(
	\us \psi_{\textrm{Om}}^\REF
	-
	\frac{\us \psi_{\Ox}^\REF}{2}
\right)
-
2 m_\textrm{e} \psi_{\textrm{e}}^\REF.
\end{align}
\end{subequations}




\section{Other variants}
Only regarding surface:

\def\surf{\mathrm{s}}
\def\Y{\mathrm{Y}}
\def\gas{\mathrm{g}}
\subsection{model\_ysz\_GAS\_exp\_ads}
Model including adsorbtion of oxygen $\mathrm{O}_2(g)$ from gas to the surface. All kinetics are exponential-like. The complete reaction mechanism reads
\begin{subequations}
\begin{align}
\mathrm{Oxide \ anion \ adorption \ from \ YSZ} \ (A) &: \quad 
\ce{\Ox^{2-}(\Y) <=> \Ox^{2-}(\surf)}
\\
\mathrm{Electron \ transfer \ reaction}\ (R) &: \quad 
\ce{\textrm{O}(\surf) +  2\textrm{e}^-(\surf) <=> \Ox^{2-}(\surf)}
\\
\mathrm{Oxygen \ adorption \ from \ gas}\  (O) &: \quad 
\ce{\Ox_2({\gas}) <=> 2 \Ox(\surf)}
\\
\end{align}
\end{subequations}

Chemical potentials reads
\begin{subequations}
\begin{align}
\mu_{\OO} &= \psi_{\OO}^\REF + \frac{\kB T}{m_\Ox^2}\ln 
\left( 
	\pO2 
\right)
\\
\mu_{\us \Ox} 
&= 
\us \psi_{\Ox}^\REF 
+ 
\frac{\kB T}{m_\Ox} \ln 
\left(  
	\frac{\yOs}{1-\yOs}
\right).
\end{align}
\end{subequations}

The surface system reads
\begin{subequations}
\begin{align}
\partial_t \frac{m_\Ox \us m (1- \us \nu)}{\aL} \ys
&=
m_\Ox \us A
+
m_\Ox \us R,
\\
\partial_t \frac{m_\Ox}{4 \aL} \yOs
&=
2 m_\Ox \us O
-
m_\Ox \us R,
\end{align}
\end{subequations}
with
\begin{subequations}
\begin{align}
&\us A
=
\frac{\A0}{\SA}
\Bigg[
	\textrm{exp}
	\left(
		{-\betaA \SA \frac{\DGA}{\kB T}}
	\right)
	\left(
		\frac{1 - \y0}{\y0}
		\frac{\ys}{1-\ys}
	\right)^{-\betaA \SA}
\\ \nonumber
	&\quad \quad \quad \quad	
	-
	\textrm{exp}
	\left(
		{(1-\betaA) \SA \frac{\DGA}{\kB T}}
	\right)
	\left(
		\frac{1 - \y0}{\y0}
		\frac{\ys}{1-\ys}
	\right)^{(1-\betaA) \SA}
\Bigg],
\\ \nonumber
\\
&\us R
=
\frac{\R0}{\SR}
\Bigg[
	\textrm{exp}
	\left(
		{-\betaR \SR \frac{\DGR}{\kB T}}
	\right)
	\left(
		\frac{\ys}{1- \ys}
		\frac{1-\yOs}{\yOs}
	\right)^{-\betaR \SR}
\\ \nonumber
	& \quad \quad \quad \quad	
	-
	\textrm{exp}
	\left(
		{(1 - \betaR) \SR \frac{\DGR}{\kB T}}
	\right)
	\left(
		\frac{\ys}{1- \ys}
		\frac{1-\yOs}{\yOs}
	\right)^{(1-\betaR) \SR}
\Bigg],
\\ \nonumber
\\
&\us O
=
\frac{\K0}{\SO}
\Bigg[
	\textrm{exp}
	\left(
		{-\betaO \SO \frac{\DGO}{\kB T}}
	\right)
	\left(
		\left(
			\frac{\yOs}{1-\yOs}
		\right)^2
		\frac{1}{\pO2}
	\right)^{-\betaO \SO}
\\ \nonumber
	&\quad \quad \quad \quad	
	-
	\textrm{exp}
	\left(
		{(1-\betaO) \SO \frac{\DGO}{\kB T}}
	\right)
	\left(
		\left(
			\frac{\yOs}{1-\yOs}
		\right)^2
		\frac{1}{\pO2}
	\right)^{(1 - \betaO) \SO}
\Bigg],
\end{align}
\end{subequations}
and we define
\begin{subequations}
\begin{align}
\DGA
&=
m_\Ox
\left(
	\psi_{\textrm{Om}}^\REF
	-
	\us \psi_{\textrm{Om}}^\REF
\right)
\\
\DGR
&= 
m_\Ox
\left(
	\us \psi_{\textrm{Om}}^\REF
	-
	\frac{\us \psi_{\Ox}^\REF}{2}
\right)
-
2 m_\textrm{e} \psi_{\textrm{e}}^\REF.
\\
\DGO
&= 
m_\Ox
\left(
	\us \psi_{\Ox}^\REF
	-
	2 \psi_{\OO}^\REF
\right).
\end{align}
\end{subequations}
The equilibrium condition is
\begin{subequations}
\begin{align}
\frac{\y0^\eq}{1-\y0^\eq} = \sqrt{\pO2} \ \textrm{exp}
\left(
	\frac{\DGA - \DGR - \frac{\DGO}{2}}{\kB T}
\right).
\end{align}
\end{subequations}
\end{document}
